\documentclass{report}
\usepackage{amsmath}
\usepackage{amsfonts}
\usepackage{geometry}
\geometry{margin=1in}

\title{ Report}
\author{Your Name}
\date{\today}

\begin{document}

\maketitle

\tableofcontents

\chapter{Introduction}

This report summarizes the results for Dispatch with Load Duration Curve and energy or CO2 constraint. The main objective was to solve economic dispatch problems under generator capacity and CO₂ emission constraints.

\chapter{Part A -- Manual Dispatch}

This section involves manually solving for generator schedules under various G1 energy constraints.

\section{Summary of cases}

\begin{table}[h]
\centering
\begin{tabular}{|c|c|c|c|c|c|c|}
\hline
 & G1 Energy Limit & G1 (GWh) & G2 (GWh) & G3 (GWh) & G4 (GWh) & Market (Price) \\
\hline
a & None & 336 & 224 & 168 & 0 & 80, 40, 30 \\
b &  273 &  273 & 238 & 169 & 0 & 80, 40, 40 \\
c &  168 &  168 & 277 & 169 & 0 & 80, 40, 40 \\
d &   63 &   63 & 353 & 169 & 0 & 80, 40, 40 \\
\hline
\end{tabular}
\caption{Dispatch under G1 energy constraints}
\end{table}

\chapter{Part B -- Model with CO₂ Constraints}

\section{Mathematical formulation}

\[
\min \sum_t \sum_g C_g p_{Ggt}
\]
with:
\[
\sum_g p_{Ggt} = PD_t \quad \forall t
\]
\[
0 \le p_{Ggt} \le P_g^+ \quad \forall g,t
\]
\[
\sum_t \sum_g z_g p_{Ggt} \le CO_2\text{ limit}
\]
with:
\[
z_g = \frac{WF_g}{E_g}
\]

\section{Model implementation (Mosel)}

The implemented Mosel script efficiently solves for pGgt, shadow prices for CO₂, capacity shadow prices, and average hourly CO₂ output.

\section{Summary of results}

\begin{table}[h]
\centering
\begin{tabular}{|c|c|c|c|c|}
\hline
 & OCGT & CCGT & Coal & Diesel \\
\hline
Effficiency & +0.47 & +0.53 & +0.38 & +0.35 \\
CO₂ per MWh & +0.39 & +0.34 & +0.83 & +0.69 \\
shadow price & +94.13 & +119.44 & +0 & +0 \\
average CO₂ (t/hr) & +1,507.5 & +5,200 & +19,000 & +992.5 \\
\hline
\end{tabular}
\caption{Summary of generator performance under CO₂ constraint}
\end{table}

\section{CO₂ Constraints (Analysis)}    

\begin{table}[h]
\centering
\begin{tabular}{|c|c|}
\hline
CO₂ Limit &  Shadow price \\
\hline
10,205.5 & +496.27 \\
12,465.5 & +  33.21 \\
17,418.8 & +  10.09 \\
above & +   0 \\
\hline
\end{tabular}
\caption{Shadow prices for CO₂ constraint}
\end{table}

\section{Summary (Graphical representation and Comments)}    

- The average hourly cost drops as CO₂ constraint is relaxed.  
- The average power output by generator increases alongside this relaxation.  
- The shadow price drops to zero when the constraint is non-binding.  

\chapter{Conclusion}

This project demonstrates:
\begin{itemize}
\item The economic algorithm for power dispatch under constraints.
\item The shadow price’s role in reflecting constraint tightness.
\item The operational policy’s adjustment in response to policy signals.
\end{itemize}

\end{document}
